% LaTeX
% Allan de Alencar Ramos
% Tables

% Preamble
\documentclass[12pt, a4paper]{article}

% Packages related to language and characters
\usepackage[T1]{fontenc}
\usepackage[utf8]{inputenc}

% Set the language of the document to Brazilian Portuguese
\usepackage[brazilian]{babel}

% Package related to paper margins
\usepackage[top = 2cm, bottom = 2cm, left = 2.5cm, right = 2.5cm]{geometry}

% Line spacing
\linespread{1.5}

\begin{document}

\title{LaTeX}
\author{Allan de Alencar Ramos}
\maketitle

% Uso dos comandos l (left), c (center) e r (right)
% O separador | acrescenta uma linha vertical
% \hline acrescenta uma linha horizontal
% Separar colunas com & e separar linhas com \\
\begin{tabular}{| l | c | r |} \hline
	Célula 1 & Célula 2 & Célula 3 \\ \hline
	Célula  4 & Célula 5 & Célula 6 \\ \hline
	Célula 7 & Célula 8 & Célula 9 \\ \hline
\end{tabular}

\vspace{2cm}

% O multicolum tem três argumentos: número, posição e item
% Todos os elementos abaixo estão centralizados
% \multicolumn{Colunas a serem mescladas}{Alinhamento}{Nome do item}
\begin{tabular}{| c | c | c | c |} \hline
	\multicolumn{4}{| c |}{Meses do ano} \\ \hline
	Janeiro & Fevereiro & Março & Abril \\ \hline
	Maio & Junho & Julho & Agosto \\ \hline
	Stembro & Outubro & Novembro & Dezembro \\ \hline
\end{tabular}

\vspace{2cm}

\begin{tabular}{| c | cc |} \hline
	Números & \multicolumn{2}{c|}{Meses do ano e abreviação} \\ \hline
	1 & Janeiro & Jan \\ \cline{2-3}
	2 & Fevereiro & Fev \\ \cline{2-3}
	3 & Março & Mar \\ \cline{2-3}
	4 & Abril & Abr \\ \cline{2-3}
	5 & Maio & Maio \\ \cline{2-3}
	6 & Junho & Jun \\ \hline
\end{tabular}

\vspace{2cm}

% O Centring centraliza a tabela na página
% Possíveis opções dentro do colchetes
% h: A tabela será mantida em um local padrão definido pelo TeX
% b: A tabela será colocada na parte inferior da página
% t: A tabela será colocada na parte superior da página
% p: A tabela será colocada em uma nova página
\begin{table}[h]
\centering
\caption{Os maiores países do mundo em extensão}
\vspace{0.5cm}
\begin{tabular}{ c | cc |}
	Posição & Países & Extensão territorial ($km^{2}$) \\ \hline
	1 & Rússia 			& 17.098.246 \\
	2 & Canadá 			& 9.984.670 \\
	3 & China 			& 9.596.961 \\
	4 & Estados Unidos 	& 9.371.174 \\
	5 & Brasil 			& 8.515.767 \\

\end{tabular}

\end{table}

\end{document}
