% LaTeX
% Allan de Alencar Ramos
% Summary and Thanks

% Preamble
\documentclass[12pt, a4paper, oneside]{book}

% Packages related to language and characters
\usepackage[T1]{fontenc}
\usepackage[utf8]{inputenc}

% Set the language of the document to Brazilian Portuguese
\usepackage[brazilian]{babel}

% Package related to paper margins
\usepackage[top = 2cm, bottom = 2cm, left = 2.5cm, right = 2.5cm]{geometry}

% Package related to figures
\usepackage{graphicx}

\usepackage{indentfirst}

% Line spacing
\linespread{1.5}

\begin{document}

\begin{titlepage}

% Espaçamento entre o top da página e o primeiro parágrafo
\addtolength{\topmargin}{1.5cm}

% Espaçamento entre um parágrafo e outro
\setlength{\baselineskip}{1.4\baselineskip}

\begin{center}
\large{NOME DA UNIVERSIDADE}

\large{INSTITUIÇÃO ACADÊMICA OU ESCOLA OU FACULDADE }
\end{center}

\vspace{2cm}

\begin{center}
\Large{\textbf{Título do seu trabalho}}
\end{center}

\vspace{1.5cm}

\begin{center}
% Author
\Large{Allan de Alencar Ramos}
\end{center}

\vspace{2cm}

\begin{flushright}
\begin{minipage}{10cm}
\hrulefill

Trabalho final de graduação do curso de Matemática da Universidade Estadual de Campinas apresentado como pré-requisito para a obtenção do grau de Bacharel em Matemática.

\hrulefill

\textbf{Orientador: Prof. Dr. Fulano}

\end{minipage}
\end{flushright}

\setlength{\baselineskip}{0.7\baselineskip}

% Com o comando \vfill é adicionado certo espaçamento que não extrapole os limites da página
\vfill

\begin{center}
São Paulo

Setembro de 2025
\end{center}

\end{titlepage}

% Lembre-se de que para utilizar o comando \chapter é necessário que o estilo do documento seja book

% Resumo
\chapter*{Resumo}
\noindent
Aqui vai o resumo do seu trabalho... \\
\textbf{Palavras-chave: } Português, Matemática, Física, Química 

% Agradecimentos
\chapter*{Agradecimentos}
\noindent
Aqui vai os agradecimentos para as pessoas mais próximas deste trabalho...

\end{document}
