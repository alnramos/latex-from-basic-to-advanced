% LaTeX
% Allan de Alencar Ramos
% System of Linear Equations

% Preamble
\documentclass[12pt, a4paper]{article}

% Packages related to language and characters
\usepackage[T1]{fontenc}
\usepackage[utf8]{inputenc}

% Set the language of the document to Brazilian Portuguese
\usepackage[brazilian]{babel}

% Package related to paper margins
\usepackage[top = 2cm, bottom = 2cm, left = 2.5cm, right = 2.5cm]{geometry}

% pmatrix package
\usepackage{amsmath, array, amssymb}

% Line spacing
\linespread{1.5}

\begin{document}

\title{LaTeX}
\author{Allan de Alencar Ramos}
\maketitle

\begin{center}
\large\textbf{Ambiente Matemático}
\end{center}
\vspace{0.5cm}

% Sistema de Equações Lineares
% Existem dois ambientes para implementar o sistema de equações: array e eqnarray

% Sempre que você utilizar o left, também é necessário utilizar o right, e vice-versa
\begin{equation}
\left\lbrace
\begin{array}{cc}
	3x + 2y = 6 \\
	2x + 3y = 5 \\
\end{array}
\right.
\end{equation}

\begin{equation}
\left\lbrace
\begin{array}{ccc}
	x + y + z = 6 \\
	x + 2y + 2z = 9 \\
	2x + y + 3z = 11 \\
\end{array}
\right.
\end{equation}

% Neste ambiente, as equações são independentes
\begin{eqnarray}
	3x + 2y = 6 \\
	2x + 3y = 5
\end{eqnarray}

% Um sistema condicional de equação linear
\begin{equation}
f(x) = 
\left\lbrace
\begin{array}{cc}
	x - 1, & x = 2 \\
	2x + 3, & x \neq 2
\end{array}
\right.
\end{equation}

\end{document}
