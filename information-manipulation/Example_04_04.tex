% LaTeX
% Allan de Alencar Ramos
% Summary of Figures and Tables

% Preamble
\documentclass[12pt, a4paper, oneside]{book}

% Packages related to language and characters
\usepackage[T1]{fontenc}
\usepackage[utf8]{inputenc}

% Set the language of the document to Brazilian Portuguese
\usepackage[brazilian]{babel}

% Package related to paper margins
\usepackage[top = 2cm, bottom = 2cm, left = 2.5cm, right = 2.5cm]{geometry}

\usepackage{graphicx}
\usepackage{subcaption}

% Line spacing
\linespread{1.5}

\begin{document}

\title{LaTeX}
\author{Allan de Alencar Ramos}
\maketitle

% Summary
\tableofcontents

% Sumary of Figures
\listoffigures

% Summary of Tables
\listoftables

\newpage

\chapter{Introdução}

\section{Assunto 1}

\begin{table}[h]\label{TabMaioresPaisesEmExtensao}
\centering
\caption{Os maiores países do mundo em extensão}
\vspace{0.5cm}
\begin{tabular}{ c | cc |}
	Posição & Países & Extensão territorial ($km^{2}$) \\ \hline
	1 & Rússia 			& 17.098.246 \\
	2 & Canadá 			& 9.984.670 \\
	3 & China 			& 9.596.961 \\
	4 & Estados Unidos 	& 9.371.174 \\
	5 & Brasil 			& 8.515.767 \\

\end{tabular}

\end{table}

\section{Assunto 2}

\chapter{Material e Métodos}

\section{Assunto 3}

\begin{figure}[h]
\centering
\subcaptionbox{Brazil\label{Brazil}}{\includegraphics[scale=0.5]{/home/allan/Pictures/LaTeX/brazil.png}}
\subcaptionbox{Argentina\label{Argentina}}{\includegraphics[scale=0.5]{/home/allan/Pictures/LaTeX/argentina.png}}
\subcaptionbox{Brazil\label{Colombia}}{\includegraphics[scale=0.5]{/home/allan/Pictures/LaTeX/colombia.png}}
\caption{Alguns países da América do Sul}\label{FigSouthAmerica}

\end{figure}

\vspace{1cm}

A Figura \ref{Brazil} representa a bandeira do Brasil, enquanto que as Figuras \ref{Argentina} e \ref{Colombia} representam as bandeiras da Argentina e Colombia respectivamente. Desta forma, a Figura \ref{FigSouthAmerica} representam três países da América do Sul.

\section{Assunto 4}

\chapter{Conclusão}

\end{document}
