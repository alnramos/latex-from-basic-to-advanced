% LaTeX
% Allan de Alencar Ramos
% Building Equations using Mathematical Symbols

% Preamble
\documentclass[12pt, a4paper]{article}

% Packages related to language and characters
\usepackage[T1]{fontenc}
\usepackage[utf8]{inputenc}

% Set the language of the document to Brazilian Portuguese
\usepackage[brazilian]{babel}

% Package related to paper margins
\usepackage[top = 2cm, bottom = 2cm, left = 2.5cm, right = 2.5cm]{geometry}

% Line spacing
\linespread{1.5}

\begin{document}

\title{LaTeX}
\author{Allan de Alencar Ramos}
\maketitle

\begin{center}
\large\textbf{Ambiente Matemático}
\end{center}
\vspace{0.5cm}

% O underline indica que o elemento subsequente será subscrito
\begin{equation}
\lim_{x \rightarrow 2} (x^{5} - 32)
\end{equation}

\begin{equation}
\lim_{x \rightarrow 3} \sqrt{x^{3} - 27}
\end{equation}

\begin{equation}
\lim_{x \rightarrow -3} \frac{(x^{2} - 9)}{(x + 3)} 
\end{equation}

\begin{equation}
\lim_{x \rightarrow \infty} \frac{1}{x} 
\end{equation}

\begin{equation}
\int (e^{-5x} + x^{-x}) dx
\end{equation}

% _{a}: limite inferior da integral
% ^{b}: limite superior da integral
\begin{equation}
\int_{a}^{b} f(x) dx = F(b) + 9 - F(a) - 9
\end{equation}

\end{document}
