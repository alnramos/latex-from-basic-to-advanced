% LaTeX
% Allan de Alencar Ramos
% Building Equations using Mathematical Symbols

% Preamble
\documentclass[12pt, a4paper]{article}

% Packages related to language and characters
\usepackage[T1]{fontenc}
\usepackage[utf8]{inputenc}

% Set the language of the document to Brazilian Portuguese
\usepackage[brazilian]{babel}

% Package related to paper margins
\usepackage[top = 2cm, bottom = 2cm, left = 2.5cm, right = 2.5cm]{geometry}

% Line spacing
\linespread{1.5}

\begin{document}

\title{LaTeX}
\author{Allan de Alencar Ramos}
\maketitle

\begin{center}
\large\textbf{Ambiente Matemático}
\end{center}
\vspace{0.5cm}

% Lei da Gravitação Universal de Newton
\begin{equation}
\vec{F} = -G \frac{m_1 m_2}{r^{2}} \hat{r}.
\end{equation}

% Valor da constante G
\begin{equation}
6,6 \times 10^{-11} \frac{m^{3}}{Kg^{-1}s^{-2}}
\end{equation}

\begin{equation}
f(t) = \frac{1}{2} + \frac{cos \frac{\pi}{3}}{2\pi} \sum_{-\infty}^{\infty}  \frac{1}{n} e^{Bn2\pi t}
\end{equation}

\begin{equation}
\frac{ a }{ b + \frac{b+1}{ c + \frac{c+1}{d + \frac{d+1}{e} } } }
\end{equation}

% Uso dos delimitadores para melhorar a visualização das expressões e equações
\begin{equation}
\left( \frac{a}{b} \right)
\end{equation}

\begin{equation}
\left[ \frac{a}{b} \right]
\end{equation}

\begin{equation}
\left\lbrace \frac{a}{b} \right\rbrace
\end{equation}

\begin{equation}
\frac{d}{dt} \left( mr^{2} \frac{d\theta}{dt} \right) = 0
\end{equation}

\end{document}
