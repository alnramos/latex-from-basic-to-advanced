% LaTeX
% Allan de Alencar Ramos
% Basic Operations

% Preamble
\documentclass[12pt, a4paper]{article}

% Packages related to language and characters
\usepackage[T1]{fontenc}
\usepackage[utf8]{inputenc}

% Set the language of the document to Brazilian Portuguese
\usepackage[brazilian]{babel}

% Package related to paper margins
\usepackage[top = 2cm, bottom = 2cm, left = 2.5cm, right = 2.5cm]{geometry}

% Line spacing
\linespread{1.5}

\begin{document}

\title{LaTeX}
\author{Allan de Alencar Ramos}
\maketitle

\begin{center}
\large\textbf{Ambiente Matemático}
\end{center}
\vspace{0.5cm}

% Soma
$ y = 4 + 5 $

% Subtração
$ y = 5 - 4 $

% Multiplicação
$ y = 4 * 5 $
$ z = 3.5 $
$ x = 8 \times 9 $

% Divisão
$ y = 10 / 5 $

% Frações
% Sintaxe: frac{numerador}{denominador}
\begin{equation}
\frac{4}{2}
\end{equation}

\begin{equation}
\frac{( 20 + 15 + 60 )}{( 5 + 1 + 4 )}
\end{equation}

% Exponenciação
\begin{equation}
a^{( 3 + 2 )}
\end{equation}

\begin{equation}
a^{\frac{5}{3}}
\end{equation}

% Radiciação
% Sintaxe: sqrt[índice]{radicando} ou sqrt{radicando}
\begin{equation}
\sqrt[3]{27}
\end{equation}

\begin{equation}
\sqrt{8}
\end{equation}

% Logaritmação
\begin{equation}
\log 7
\end{equation}

% Log de 100 na base 10
% Sintaxe: log_{base} logaritmando
\begin{equation}
\log_{10} 100
\end{equation}

\end{document}
