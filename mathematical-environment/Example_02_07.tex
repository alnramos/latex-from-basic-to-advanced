% LaTeX
% Allan de Alencar Ramos
% Matrix

% Preamble
\documentclass[12pt, a4paper]{article}

% Packages related to language and characters
\usepackage[T1]{fontenc}
\usepackage[utf8]{inputenc}

% Set the language of the document to Brazilian Portuguese
\usepackage[brazilian]{babel}

% Package related to paper margins
\usepackage[top = 2cm, bottom = 2cm, left = 2.5cm, right = 2.5cm]{geometry}

% pmatrix package
\usepackage{amsmath, array, amssymb}

% Line spacing
\linespread{1.5}

\begin{document}

\title{LaTeX}
\author{Allan de Alencar Ramos}
\maketitle

\begin{center}
\large\textbf{Ambiente Matemático}
\end{center}
\vspace{0.5cm}

% Existem dois ambientes para realizar a implementação de matrizes: array e pmatrix

% Ao utilizarmos o ambiente array, o posicionamento do elemento é especificado por letras 
% As linhas são separadas por (\\) e as colunas por (&)

% Matriz quadrada de ordem 2 (2x2)
\begin{equation}
\left(
\begin{array}{lr}
	a & b \\
	c & d \\
\end{array}
\right)
\end{equation}

% Matriz de ordem três (3x3)
\begin{equation}
\left(
\begin{array}{lcr}
	a & b & c\\
	d & e &  f \\
	g & h & i \\
\end{array}
\right)
\end{equation}

% Matriz 2x5
\begin{equation}
\left(
\begin{array}{llcrr}
	a & b & c & d & e \\
	f & g & h & i & j \\
\end{array}
\right)
\end{equation}

% Ambiente pmatrix
% É o ambiente recomendado pois ser mais flexível

\begin{equation}
\begin{pmatrix}
	x & y & x \\
	w & h & r \\
\end{pmatrix}
\end{equation}

\begin{equation}
\begin{pmatrix}
	x & y & z & i \\
	w & h & r & h \\
	r & t & j & g \\
\end{pmatrix}
\end{equation}

\end{document}
